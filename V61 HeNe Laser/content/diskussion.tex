\section{Diskussion}
\label{sec:Diskussion}

Da die Lasertätigkeit sehr empfindlich auf kleinste Verrückungen des Aufbaus reagiert, konnte nur für eine Spiegelanordnung ein Laserstrahl aufgebaut werden.
Die für diese Anordnung gemessene maximale Resonatorlänge von \SI{134}{\centi\meter} liegt relativ nah an der Resonatorlänge von \SI{140}{\centi\meter}, die die untere Schranke der Stabilitätsbedingung gerade noch erfüllt.

Für die Grundmode TEM$_{00}$ lassen sich die Messwerte mit sehr geringen Unsicherheiten durch  die Theoriekurve beschreiben.
Die Vermessung der Moden TEM$_{01}$ und TEM$_{02}$ zeigen jedoch eine starke Asymmetrie auf, welche durch die angenommene Theoriekurven der Intensitäten nicht beschrieben werden.
Dies könnte auf eine ungenaue Projektion der einzlenen Moden zurückzuführen sein, die neben der gewünschten Mode weitere ungwollte Moden enthält.

Für die Polarisation wird eine Periodizität von $2\pi$ erwartet, das Licht ist also linear polarisiert.
Die Messung bestätigt diese Annahme.
Bei der Vermessung der transversalen Moden und der Polarisation stellt die Fluktuation der Laserintensität eine Fehlerquelle dar.
Diese wird durch Verrückungen am Aufbau, sowie das nicht vollständig konstante Restlicht im Versuchsraum hervorgerufen.
Durch das Wechseln der Größenordnungen des Ampermeters können weitere systematische Fehler aufgetreten sein.

Für die drei vermessenen Resonatorlängen liegen die mittleren Frequenzdifferenzen im MHz Bereich.
Damit liegen die Abstäde der Moden weit unterhalb der Aufweitung der Frequenzen durch den Dopplereffekt, welcher im GHz bereich liegt.
Die longitudinalen Moden treten als Schwebung im Multimodenbetrieb des Lasers auf.
Der lineare Zusammenhang zwischen Resonatorlänge und Wellenlängendifferenz konnte gezeigt werden.

Die mit dem Experiment bestimmte Wellenlänge ist $\lambda = \num{631.29(539)}$ nm.
Die theoretische Wellenlänge ist $\lambda = 632,8$ nm.
Damit passt die bestimmte Wellenlänge im Bereich der Fehlertoleranz zum Theoriewert.
Als mögliche Fehlerquelle kommt die Vermessung der Abstände $L$ und $d_n$ in Frage und in Verbindung damit die ungenaue Auflösung der Nebenmaxima der Intensität.
