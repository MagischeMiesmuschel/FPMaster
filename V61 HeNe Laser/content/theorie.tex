\section{Ziel}
Am Beispiel des Helium-Neon-Lasers sollen die grundlegenden Konzepte von Lasern erarbeitet werden.
Diese umfassen deren Stabilität, transversale Moden, die Polaristaion des Laserlichts und der Wellenlänge

\section{Theorie}
\label{sec:Theorie}

Laserstrahlen sind elektromagnetische Wellen mit einer hohen Intensität.
Sie sind stark monochromatisch und besitzen eine große Kohärenzlänge.
In einer Laserapperatur, bestehend aus drei Hauptbestandteilen, werden diese erzeugt.
Die drei Bestandteile sind ein aktives Medium und eine Pumpe die einen Strahl erzeugen, welcher durch zwei Spiegel, die den Resonator formen, total und teilreflektiert wird.
Durch den öffteren Durchlauf durch das aktive Medium ist eine große Verstärkung möglich.

\subsection{Erzeugung von Laserlicht}

Die drei wichtigsten Wechselwirkungen sind die Absoption von Photonen, die spontane Emission und die stimmulierte Emission.\\
Bei der Absorption wird ein Photon von einem Atom absorbiert, womit ein Elektron in ein angeregtes Niveau gehoben wird.
Seine Energie steigt und das Photon verschwindet.
Dieser Prozess kann nur stattfinden, wenn $E_\gamma \geq \Delta E_{Niveau}$ gilt.\\
Die spontane Emission ist der typische Zerfall von angeregten Zuständen mit einer begrenzten Lebensdauer in Zustände mit einer niedrigeren Energie.
Dabei wird ein Photon ausgesendet, dessen Energie der Energiedifferenz der Niveaus $\Delta E_{Niveau}$ entspricht.\\
Bei der stimmulierte Emission löst ein Photon mit der Energie der Energiedifferenz den Zerfall aus.
Dabei wird ein dem einfallenden Photon in Richtung, Phase und Enerie gleiches Photon freigesetzt.
Somit sind die beiden Photonen kohärent.\\
Nur der Mechanismus der stimmulierten Emission ermöglicht eine Verstärkung für kohärentes, monochromatisches Licht.
Dafür muss eine Besetzungsinversion auftreten in den Niveaus des aktiven Mediums, indem ein Zustand, der größer als der Grundzustand ist, mehr als zur Hälfte gefüllt ist.
Dies st nicht für ein Zweiniveausystem zu realisieren.
Dort entspricht die Energie für Anregung der für die stimmulierte Emission und omit konkurrieren diese Prozesse.
Somit ist ein System mit mindestens drei Niveaus nötig.
Elektronen werden aus dem Grundzustand in einen hoch angeregten Zustand gehoben, welcher schnell durch spontane Emission in einen metastabilen Zustand zerfällt mit höherer Lebensdauer.
Bei He-Ne-Laser geschiet dies durch Stöße zweiter Art zwischen den Helium- und Neonatomen.
Nun können Photonen mit passender Energie die stimmulierte Emission auslösen, welche anfänglich aus der spontanen Emission des metastabilen Zustands kommen.\\
Durch die Spiegel werden die Photonen aus der stimmulierten Emission öfters durch das Medium geführt um eine lawinenartige Vervielfälltigung zu erzeugen.

\subsection{Stabilität}

Die Bedingung
\begin{equation}
    0 \le g_1g_2 \le 1
    \label{eqn:stabi}
\end{equation}
muss für die Stabilitätsparameter $g_i =1-\frac{L}{r_i}$ gelten, damit der Resonator optisch stabil ist.
$L$ ist der Abstand beider Spiegel und $r_i$ der Krümmungsradius.
Bei einem stabilen Resonator sind die Verluste kleiner als die Verstärkung durch die stimmulierte Emission.

\subsection{Lasermoden}

Weil die Resonatorlänge $L$ sehr groß zur Wellenlänge $\lambda$ des Lasers ist, sind viele Moden möglich.
Mit $l$ und $p$ werden die Knoten in $x-$ und $y-$Richtung genannt.
Die Eigenschwingungen werden als TEM$_{l,p}$ bezeichnet, wobei die gaußförmige Grundmode TEM$_{0,0}$ den größten Anteil am Modenspektrum besitzt, weil diese weniger Verluste als höhere Moden aufweist.