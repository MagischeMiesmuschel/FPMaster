\section{Auswertung}
\label{sec:Auswertung}

Im folgenden Abschnitt werden die gemessenen Daten visualisiert und ausgwertet.
Die Rechnungen und Ausgleichsrechnungen werden mit python und verschiedner Bibliotheken durchgeführt.

\subsection{Magnetfeldmessung}
\label{sec:magFeld}

Zur Bestimmung des maximalen Magnetfeldes innerhalb des Elektromagneten werden die gemessenen Werte aus Tabelle \ref{tab:magFeld} in Abbildung \ref{abb:magFeld} dargetsellt.
Es wird die Verschiebung auf der Symmetrieachse gegen die Magnetfeldstärke abgebildet.
Für die alle weiteren Rechnungen wird der maximale gemessene Wert
\begin{equation}
  B_{\text{max}}=409\,\text{mT}
\end{equation}
als vorherrschendes Magnetfeld an der Position der Probe angenommen.
\begin{figure}
  \centering
  \includegraphics{bfeld.pdf}
  \caption{Zusammenhang zwischen Magnetfeldstärke und Abstand auf der Symmetrieachse.}
  \label{abb:magFeld}
\end{figure}
\FloatBarrier

\begin{table}
  \centering
  \caption{Magnetfeldmessung.}
  \label{tab:magFeld}
  \begin{tabular}{c | c}
    \toprule
    Position in mm & Magnetfeldstärke in mT \\
    \midrule
    0.0 & 251.0 \\
    1.0 & 319.0 \\
    2.0 & 356.0 \\
    3.0 & 376.0 \\
    4.0 & 388.0 \\
    5.0 & 396.0 \\
    6.0 & 402.0 \\
    7.0 & 405.0 \\
    8.0 & 408.0 \\
    9.0 & 409.0 \\
    10.0 & 409.0 \\
    11.0 & 407.0 \\
    12.0 & 407.0 \\
    13.0 & 400.0 \\
    14.0 & 393.0 \\
    15.0 & 383.0 \\
    16.0 & 372.0 \\
    17.0 & 353.0 \\
    18.0 & 328.0 \\
    19.0 & 295.0 \\
    20.0 & 254.0 \\
    \bottomrule
  \end{tabular}
\end{table}
\FloatBarrier

\subsection{Messung mit undotiertem GaAs}
\label{sec:undotiert}

Die Messung mit der undotierten GaAs-Probe liefert die Ergebnisse in Tabelle \ref{tab:undot}.
Mit der Dicke der Probe von $L=5.11\,$mm und den gemessenen Werten wird ein normierter Wert der Faradayrotation bestimmt und in die Tabelle mit den Messwerten getragen.
Diese normierten Werte werden gegen $\lambda^2$ aufgetragen und sind in Abbildung \ref{abb:undot} zu sehen.
Es ist ein deutlich abfallender Trend mit zunehmender Wellenlänge zu erkennen.

\begin{table}
  \centering
  \caption{Undotierte GaAs-Probe und normierte Faradayrotation.}
  \label{tab:undot}
  \begin{tabular}{c | c | c | c}
    \toprule
    $\lambda$/$\mu$m & $\Theta(+B)$/° & $\Theta(-B)$/°& $\Theta_{\text{norm}}$/rad/mm \\
    \midrule
    1.06 & 267.0 & 243.0 & 0.082 \\
    1.29 & 261.0 & 248.5 & 0.043 \\
    1.45 & 261.2 & 250.0 & 0.038 \\
    1.72 & 258.0 & 249.9 & 0.028 \\
    1.96 & 252.0 & 244.8 & 0.025 \\
    2.156 & 248.7 & 243.5 & 0.018 \\
    2.34 & 225.6 & 221.5 & 0.014 \\
    2.51 & 212.0 & 208.4 & 0.012 \\
    2.65 & 177.0 & 173.4 & 0.012 \\
    \bottomrule
  \end{tabular}
\end{table}
\FloatBarrier

\begin{figure}
  \centering
  \includegraphics{undotiert.pdf}
  \caption{Grafische Darstellung der Messwerte aus Tabelle \ref{tab:undot}.}
  \label{abb:undot}
\end{figure}
\FloatBarrier

\subsection{Messung mit dotierten GaAs-Proben}
\label{sec:dot}

Die unterschiedlich dotierten Proben haben eine Konzentration von $N_{\text{leicht}}=1.2\cdot10^{18}\text{cm}^{-3}$ und $N_{\text{hoch}}=2.8\cdot10^{18}\text{cm}^{-3}$ und die Dicken $L_{\text{leicht}}=1.296\,$mm und $L_{\text{hoch}}=1.36\,$mm.
Deren Messwerte sind in den Tabellen \ref{tab:hoch} und \ref{tab:tief} aufgeführt.
Die Faradayrotation wird jeweils normiert und von diesem Wert für alle Wellenlängen die normierte Faradayrotation der undotierten Probe abgezogen.
So kann die Faradayrotation $\Delta\theta_\text{norm}$, die durch Leitungselektronen bedingt ist, untersucht werden. 
Die berechneten Werte sind in Abbildung \ref{abb:delta} grafisch dargetsellt.
In beiden Fällen ist eine lineare Abhängigkeit zu $\lambda^2$ ersichtlich, wie mit Gleichung \ref{eq:drehwinkel} erwartet wird.
Bei der leicht dotierten Probe fällt ein Wert weit aus der Reihe und wird nicht für die lineare Ausgleichrechnung berücksichtigt.
Ein Fit der Daten wird mit der Funktion
\begin{equation}
  \Delta\theta_{\text{norm}} = A\lambda^2
\end{equation}
durchgeführt und ergibt für die beiden Proben folgende Parameter:
\begin{align*}
  A_{\text{leicht}} &= (0.0170526 +/- 0.0000031) \frac{\text{rad}}{\text{mm}\mu m^2} \\
  A_{\text{hoch}} &= (0.0283207 \pm 0.0000036) \frac{\text{rad}}{\text{mm}\mu m^2}
\end{align*}
Aus der Steigung $A$ lässt sich mit Gleichung \ref{eq:drehwinkel} die effektive Mase bestimmen.
Für den Brechungsindex in GaAs wird nach \cite{nGaAs} $n=3.4$ genutzt.
Mit
\begin{equation}
  m*=\sqrt{\frac{e^3\cdot N \cdot B}{A\cdot 8\pi^2\epsilon_0 c^3 \cdot n }}
\end{equation}
ergeben sich die für die effektive Masse mit $m_e = 5.485799\cdot 10^{-32}\,$kg folgende Werte:
\begin{align*}
  m^*_{\text{leicht}} &= (4.2992 \pm 0.0004)\cdot 10^{-32}\text{kg} \\
  \frac{m*_{\text{leicht}}}{m_e} &= 0.078369\pm 0.000007 \\
  m^*_{\text{hoch}} &= (5.09582\pm 0.00032)\cdot 10^{-32}\text{kg} \\
  \frac{m*_{\text{hoch}}}{m_e} &= 0.092891\pm 0.000006\\
\end{align*}

\begin{table}
  \centering
  \caption{Leicht dotierte GaAs-Probe und normierte Faradayrotation.}
  \label{tab:tief}
  \begin{tabular}{c | c | c | c}
    \toprule
    $\lambda$/$\mu$m & $\Theta(+B)$/° & $\Theta(-B)$/°& $\Theta_{\text{norm}}$/rad/mm \\
    \midrule
    1.06 & 260.0 & 251.9 & 0.022 \\
    1.29 & 259.0 & 251.0 & 0.060 \\
    1.45 & 258.7 & 252.5 & 0.041 \\
    1.72 & 257.0 & 250.0 & 0.062 \\
    1.96 & 252.0 & 244.3 & 0.074 \\
    2.156 & 250.1 & 240.7 & 0.103 \\
    2.34 & 227.3 & 219.1 & 0.091 \\
    2.51 & 206.9 & 206.0 & -0.001 \\
    2.65 & 242.7 & 235.0 & 0.087 \\
    \bottomrule
  \end{tabular}
\end{table}
\FloatBarrier

\begin{table}
  \centering
  \caption{Hoch dotierte GaAs-Probe und normierte Faradayrotation.}
  \label{tab:hoch}
  \begin{tabular}{c | c | c | c}
    \toprule
    $\lambda$/$\mu$m & $\Theta(+B)$/° & $\Theta(-B)$/°& $\Theta_{\text{norm}}$/rad/mm \\
    \midrule
    1.06 & 161.0 & 149.5 & 0.073 \\
    1.29 & 160.5 & 151.9 & 0.073 \\
    1.45 & 159.7 & 151.1 & 0.078 \\
    1.72 & 163.0 & 154.0 & 0.094 \\
    1.96 & 170.1 & 159.5 & 0.118 \\
    2.156 & 171.0 & 161.7 & 0.107 \\
    2.34 & 194.3 & 181.3 & 0.161 \\
    2.51 & 209.5 & 193.9 & 0.198 \\
    2.65 & 247.0 & 233.8 & 0.165 \\
    \bottomrule
  \end{tabular}
\end{table}
\FloatBarrier

\begin{figure}
  \centering
  \includegraphics{deltaTheta.pdf}
  \caption{Grafische Darstellung der Messwerte aus Tabellen \ref{tab:tief} und \ref{tab:hoch}.}
  \label{abb:delta}
\end{figure}
\FloatBarrier