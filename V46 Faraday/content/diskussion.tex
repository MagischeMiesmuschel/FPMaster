\section{Diskussion}
\label{sec:Diskussion}
Der erwartete lineare Zusammenhang der Faradayrotation und dem quadrat der Wellenlänge des Lichtes konnte in diesem Versuch gezeigt werden (siehe Abbildung \ref{abb:delta}).
Dadurch wird das Konzept der effektiven Masse und das semiklassische Modell für freie Ladungsträger in Halbleitern experimentell bestätigt.
\\
Die berechneten Werte für die effektive Masse von $m^*_{\text{leicht}}(0.047195\pm 0.000004)\,m_e$ und $m*_{\text{hoch}}(0.055940+\pm 0.000004)\,m_e$ 
liegen im Vergleich mit dem Theoriewert von $m^*_{\text{eff}}=0.067m_e$ \cite{meff} um $29.56\%$ und $16.51\%$ drüber.
Somit liegen beide Werte in der Nähe des Theoriewerts, aber die immernoch große Diskrepanz kann aus dem Messverfahren kommen.
Es war nicht möglich einen Nullstrom zu messen und das Minimum am Oszilloskop war oftmals nicht exakt auf eine genaue Winkeleinstellung festzulegen, sondern innerhalb eines kleinen Winkelbereichs konstant.
Ein weiteres Problem ist die abnehmende Signalstärke hin zu größeren Wellenlängen, welche die Winkelmessung erschwert und zu dem Fehlpunkt führen könnte.
Trotzdem kann der Aufbau als geeignetes Instrument zur Bestimmung der effektiven Masse gewertet werden.