\section{Durchführung}
\label{sec:Durchführung}

Zu Beginn der Kalibrierung wird überprüft, ob das von den Sammellinsen
fokussierte Licht auf den Sensitivbereich der jeweiligen Photodioden fällt.
Anschließend werden die Photodioden mit dem Selektivverstärker verbunden.
Die Verstärkerfrequenz des Selektivverstärkers wird nun solange variiert,
bis eine maximale Signalamplitude auf dem Oszilloskop eingestellt wird.

Die Bestimmung des Polarisationswinkel $\vartheta$ erfolgt mit Hilfe eines
an dem ersten Glan-Thompson-Prisma angebrachten Goniometer. Nachdem Einsetzen einer
Probe und eines Filters, wird das Goniometer so lange variiert, bis ein Signalminimum
auf dem Oszilloskop gefunden wird. Der am Goniometer eingestellte Winkel $\vartheta_1$ wird
notiert. Eine Änderung des Magnetfeldes um $2B$ kann durch eine Umpolung des
Magnetfeldes bewirkt werden. Nach der Umpolung wird ein weiteres Mal mit dem
Goniometer ein Minimum gesucht und der dazugehörige Winkel $\vartheta_2$ notiert.
Aus den beiden notierten Winkeln ist es nun möglich, den Drehwinkel der Polarisationsebene
$\vartheta$ über die Relation
\begin{equation}
  \label{eq:theta_aus_messung}
  \vartheta = \frac{1}{2}(\vartheta_1 - \vartheta_2)
\end{equation}
zu bestimmen. Die Winkel $\vartheta_1$ und $\vartheta_2$ werden nun für verschiedene
Interferenzfilter und damit für verschiedene Wellenlängen bestimmt.
Nach der Vermessung aller Proben ist es möglich die effektive Masse der Donatorelektronen zu bestimmen.
Zum Schluss wird mit Hilfe einer Hallsonde das Magnetfeld im Inneren der Spule
vermessen.
