\section{Diskussion}
\label{sec:Diskussion}
Die vertikale Komponente des Erdmagnetfeldes wir zu $B_{vert}=30.65\,\mu T$ bestimmt 
und weicht somit ungefähr um $32\,\%$ vom Literaturwert $B_{vert,lit}=45.07\,\mu T$ ab.
Aus der Regressionsgeraden für das Magnetfeld gegen die RF-Frequenz ergibt sich das horizontale Erdmagnetfeld als y-Achsen Abschnitt mit $B_{vert}=5.2\,\mu T$.
Dieser Wert weicht um $73\,\%$ vom Literaturwert $B_{hor,lit}=19.33\,\mu T$ ab.
Mögliche Fehlerquellen sind Störfelder im Raum.
\\
Die Kernspins der Rubidiumisotope wurden zu $I_{87} = 1.38\pm0.09$ und $I_{85} = 2.27\pm0.08$ berechnet.
Damit liegen die theoretischen Werte $I_{87,theo} = 1.5$ und $I_{85,theo} = 2.5$ innerhalb von zwei und drei Standardabweichungen.
Die Messung ist folglich als präzise einzuschätzen.
\\
Das Isotropenverhältnis innerhalb der Dampfzelle wird zu $\frac{N\left(^{85}\text{Rb}\right)}{N\left(^{87}\text{Rb}\right)}=2\pm0.75$ berechnet.
Bei einem natürlichen Verhältnis von $0.67$ bedeutet es, dass die uns vorliegende Probe mit $^{85}\text{Rb}$ angereichert wurde.
\\
Der vergleich des linearen und quadratischen Zeman-Effekts liefert einen Unterschied von drei Größenordnungen.
Somit kann der quadratische Teil in den Berechnungen vernachlässigt werden.
\\
Das zuletzt bestimmte Verhältnis der Parameter aus der angepassten Hyperbelfunktion ergibt sich zu $\frac{b_85}{b_87}=1.35\pm0.08$ und liegt innerhalb von zwei Standardabweichungen zum theoretischen Wert mit $1.5$. 