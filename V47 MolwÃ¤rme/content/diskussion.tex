\section{Diskussion}
\label{sec:Diskussion}

Die Angleichung der Temperaturen von Probe und Probengehäuse ist notwendig, um Verluste durch Wärmestrahlung zu unterdrücken.
Da das Heizsystem des Gehäuses nicht an die Probenheizung angepasst arbeitet,
muss es von Hand eingestellt und gegebenenfalls nachjustiert werden.
Weiterhin reagiert das System träge auf Leistungsanpassungen.
Es konnte daher die obige Forderung nicht mit Sicherheit gewährleistet werden und
somit ist von störenden Effekten, die das Ergebnis beeinflussen, auszugehen.
Außerdem kann der Austausch von Wärme durch Konvektion nicht ganz ausgeschlossen werden, da im Rezipienten kein totales Vakuum herrscht.

Für die Debye-Temperatur wurden für Kupfer der angegebenen Masse folgende Werte gefunden.
\begin{align}
	\theta_{D,\text{ Messung}} &= \SI{285.22\pm0.12}{\kelvin},\\
	\theta_{D,\text{ Th.}} &= \SI{332.48}{\kelvin}
\end{align}
Mit dem Literaturwert \cite{debye-kupfer}
\begin{equation}
	\theta_{D,\text{ Lit.}} = \SI{345}{\kelvin}
\end{equation}
ergeben sich die prozentualen Abweichungen
\begin{align}
	\symup\Delta\theta_D &= \SI{17.33}{\percent},\\
	\symup\Delta\theta_{D,\text{ Th.}} &= \SI{3.63}{\percent}.
\end{align}
Dies sind vertretbare Abweichungen im Rahmen einer nicht aufwendig vorbereiteten Messung,
obwohl der Literaturwert nicht in der einfachen Standardabweichung von $\theta_D$ liegt.

Die Molwärme ist keine in Standardwerken aufgeführte Größe, da sie von der Probenmenge abhängt.
Die spezifische Wärmekapazität ist eine Materialkonstante und wird als Molwärme pro Masse definiert.
Die spezifische Wärmekapazität ist wie die Molwärme von der Temperatur abhängig und wird in Tabellenwerken bei fester Temperatur angegeben.
Der Vergleich der experimentell bestimmten Molwärme erfolgt mit der spezifischen Wärmekapazität und der molaren Masse $M$ zu
\begin{align}
	C_{p, \, T > \SI{260}{\kelvin}} =\SI{25.35(35)}{\joule\per\kelvin\mol}\\
	C_{p,\,\text{Lit.}} = c_{p,\,\text{Lit.}}\cdot M_{\text{Kupfer}}= \SI{24.45}{\joule\per\kelvin\mol},\\
\end{align}
bei Raumtemperatur \cite{chem}, woraus sich die Abweichung
\begin{equation}
	\symup\Delta C_p = \SI{3.68}{\percent}
\end{equation}
ergibt.
Die geringe Abweichung lässt auf ein zufriedenstellenden Aufbau und Durchführung des Experimentes schließen.
Mit den Ergebnissen des Experimentes, aufbauend auf die Debye-Theorie, kann diese als gute Näherung erkannt werden.
