\section{Auswertung}
\label{sec:Auswertung}

\subsection{Untergrund}
\label{sec:Unter}
Für die Bestimmung des Untergrunds von $j(T)$ werden die Messdaten außerhalb von \SIrange{230}{240}{\kelvin} herangezogen.
Diese sind in den Abbildungen \ref{fig:plot1} und \ref{fig:plot2} blau dargestellt.
Desweiteren ist auch der Untergrundfit der Form
\begin{equation}
    \label{eqn:exp}
    I(T) = a \cdot T + b
\end{equation}
in den Abbildungen \ref{fig:plot1} und \ref{fig:plot2} eingezeichnet.
Die Parameter für die Heizrate 1 von etwa \SI{1.43}{\kelvin\per\minute} ergeben sich zu
\begin{align*}
    a_1 &= \SI{0.070(2)e-12}{\ampere\per\kelvin} \\
    b_1 &= \SI{-15.1(6)e-12}{\ampere}.
\end{align*}
und für die Heizrate 2 von etwa \SI{1.91}{\kelvin\per\minute} zu
\begin{align*}
  a_2 &= \SI{0.100(3)e-12}{\ampere\per\kelvin} \\
  b_2 &= \SI{-21.9(8)e-12}{\ampere}.
\end{align*}
Nach der Subtraktion des Untergrunds, sind einige Messwerte kleiner als Null.
Die bereinigten Daten werden um diesen Offset (in den Abbildungen \ref{fig:plot1} und \ref{fig:plot2} gelb) korrgiert.

\begin{figure}
  \centering
  \includegraphics{plot1.pdf}
  \caption{Die Messdaten für eine Heizrate 1. Nicht verwendete Messdaten sind grün und die für den Untergrund verwendeten blau dargestellt. Die bereinigten Daten sind rot, der Offset gelb.}
  \label{fig:plot1}
\end{figure}
\begin{figure}
  \centering
  \includegraphics{plot2.pdf}
  \caption{Die Messdaten für eine Heizrate 2. Nicht verwendete Messdaten sind grün und die für den Untergrund verwendeten blau dargestellt. Die bereinigten Daten sind rot, der Offset gelb.}
  \label{fig:plot2}
\end{figure}
\FloatBarrier

\subsection{Stromdichteansatz}
\label{sec:klT}
Zur Berechnung der Aktivierungsenergie $W$ bei kleinen Temperaturen werden die Daten im Bereich von \SIrange{236}{255}{\kelvin} für die erste Messung verwendet, für die zweite Messung im Bereich von \SIrange{236}{255}{\kelvin}.
Diese Bereiche sind in der halblogarithmischen Darstellung annähenrd linear.
Es wird die Näherung der Stromdichte für kleine Temperaturen herangezogen.
Der gemessene Strom $j(T)$ ist dabei bis auf einen konstanten Faktor proportional zur Stromdichte.
Der Strom wird logarithmiert und in Abhängigkeit von der reziproken Temperatur $\frac{1}{T}$ aufgetragen.
Für jede Heizrate erfolgt eine lineare Ausgleichsrechnung der Form
\begin{equation}
    \label{eqn:lin}
    \ln\left(\frac{j(T)}{\SI{1e-11}{\ampere}}\right) = -\frac{W}{\symup{k_B} T} + c,
\end{equation}
welche in Abbildung \ref{fig:plot3} für Heizrate 1 und in Abbildung \ref{fig:plot3} für Heizrate 2 abgebildet sind.
Somit lassen sich die Aktivierungsenergien $W_\text{i} = -a_\text{i} \cdot \symup{k_B}$ und Konstanten zu
\begin{align*}
    W_1 &= \SI{0.95(4)}{\electronvolt} \\
    W_2 &= \SI{1.15(5)}{\electronvolt} \\
    c_1 &= \num{18.0(18)} \\
    c_2 &= \num{27.4(22)}
\end{align*}
bestimmen.

\begin{figure}
  \centering
  \includegraphics{plot3.pdf}
  \caption{Die verwendeten Messdaten bei kleinen Temperaturen für den Stromdichteansatz bei Heizrate 1 und der dazugehörigen Ausgleichsgeraden.}
  \label{fig:plot3}
\end{figure}
\begin{figure}
  \centering
  \includegraphics{plot4.pdf}
  \caption{Die verwendeten Messdaten bei kleinen Temperaturen für den Stromdichteansatz bei Heizrate 2 und der dazugehörigen Ausgleichsgeraden.}
  \label{fig:plot4}
\end{figure}
\FloatBarrier

\subsection{Polarisationsansatz}
\label{sec:grT}
Nun wird die Aktivierungsenergie auf Grundlage des Polarisationsansatzes bestimmt.
Hierzu werden die Daten im Bereich von \SIrange{240}{265}{\kelvin} für die erste Messung verwendet, für die zweite Messung im Bereich von \SIrange{240}{276}{\kelvin}.
Dazu wird eine numerische Integration der Form
\begin{equation}
    \label{eqn:int}
    \int_T^{T^*} j(T') \symup{d}\,T'
\end{equation}
durchgeführt. Dies erfolgt mit \textit{Scientific Python} \cite{scipy} und der Trapez-Regel. Dabei ist zu beachten, dass $j(T^*) \approx \SI{0}{\ampere}$ gelten muss.
Die lässt sich für die Heizrate 1 auf $T^*_1 = \SI{275.85}{\kelvin}$ und für die Heizrate 2 auf $T^*_2 = \SI{278.45}{\kelvin}$ approximieren, da diese den kleinsten absoluten Abstand zu \SI{0}{\ampere} haben.
Wird nun
\begin{equation*}
    \ln\left(\frac{\int_T^{T^*} j(T') \symup{d}\,T'}{h_i \cdot j(T)}\right)
\end{equation*}
gegen $\frac{1}{T}$ aufgetragen und eine lineare Ausgleichrechnung der Form
\begin{equation}
    \label{eqn:lin2}
    \ln\left(\frac{\int_T^{T^*} j(T') \symup{d}\,T'}{b_i \cdot j(T)}\right) = \frac{W}{\symup{k_B} \cdot T} + \ln(\tau_0) = \frac{W}{\symup{k_B} \cdot T} + d
\end{equation}
durchgeführt, lässt sich die Aktivierungsenergie und theoretisch aus dem Parameter $d$ die Relaxationszeit $\tau_0$ bestimmen, was hier auf Grund des unbekannten Offsets nicht möglich ist.
Die in Abbildung \ref{fig:plot5} gezeigte Ausgleichsgerade für Heizrate 1 ergibt die Parameter
\begin{align*}
    W_1 &= \SI{1.114(33)}{\electronvolt} \\
    d_1 &= \num{-48.8(15)},
\end{align*}
Für die Heizrate 2 (siehe Abbildung \ref{fig:plot6}) ergeben sich diese Werte zu
\begin{align*}
    W_2 &= \SI{0.985(14)}{\electronvolt} \\
    d_2 &= \num{-42.1(6)}.
\end{align*}

\begin{figure}
  \centering
  \includegraphics{plot5.pdf}
  \caption{Werte zur Bestimmung der Aktivierungsenergie nach dem Polarisationsansatz für die Heizrate 1.}
  \label{fig:plot5}
\end{figure}
\begin{figure}
  \centering
  \includegraphics{plot6.pdf}
  \caption{Werte zur Bestimmung der Aktivierungsenergie nach dem Polarisationsansatz für die Heizrate 2.}
  \label{fig:plot6}
\end{figure}
\FloatBarrier


\subsection{Relaxationszeit}
\label{sec:relax}
Des Weiteren lässt sich aus dem Maximum von $j(T)$ auch die Relaxationszeit $\tau_0$
bestimmen:
\begin{equation}
    \label{eqn:tau0}
    \tau_0 = \frac{\symup{k_B}T_\text{max}^2}{W \cdot h} \exp\left(-\frac{W}{\symup{k_B}T_\text{max}} \right).
\end{equation}
Dabei wird für $T_\text{max}$ das Maximum der jeweiligen Heizrate nach Abzug des Untergrunds bestimmt.
Die Heizraten werden aus dem Mittelwert der mittleren Änderungsrate zwischen zwei Datenpunkten bestimmt.
Wie bereits erwähnt ist Heizrate \SI{1.43}{\kelvin\per\minute} und Heizrate \SI{1.91}{\kelvin\per\minute}.
Die Relaxationszeit wird für alle vier berechneten Aktivierungsenergien einzeln bestimmt.
\begin{align*}
  \tau_0(\text{Heizrate 1, Stromdichte}) &= \SI{1.1(19)e-18}{\second} \\
  \tau_0(\text{Heizrate 1, Polarisation}) &= \SI{5(8)e-22}{\second} \\
  \tau_0(\text{Heizrate 2, Stromdichte}) &= \SI{1.1(23)e-22}{\second} \\
  \tau_0(\text{Heizrate 2, Polarisation}) &= \SI{2.0(13)e-19}{\second} \\
\end{align*}
%Der Fehler ergibt sich dabei nach
%\begin{align*}
%    \label{eqn:FehlerTau}
%    \Delta \tau_0 & = \sqrt{\left(\frac{\symup{k_B}}{W h} \exp\left(-\frac{W}{\symup{k_B} T_\text{max}}\right) \left(\frac{W}{\symup{k_B}} + 2 T_\text{max}\right) \Delta T_\text{max} \right)^2} \\
%    &\overline{+ \left(\frac{\symup{k_B} T_\text{max}^2}{h} \exp\left(-\frac{W}{\symup{k_B} T_\text{max}}\right)\left(\frac{W}{\symup{k_B}T_\text{max}} + 1 \right) \Delta W \right)^2} \\
%    &\overline{+ \left(\frac{\symup{k_B} T_\text{max}^2}{h} \exp\left(-\frac{W}{\symup{k_B} T_\text{max}}\right)\Delta b \right)^2}.
%\end{align*}
%Zu beachten ist, dass kein Fehler für $T_\text{max}$ angenommen wurde, womit der Teil mit $\Delta T_\text{max}$ in der Gauß'schen Fehlerfortpflanzung enfällt.
