\section{Diskussion}
\label{sec:Diskussion}

Für die Bereinigung vom Untergrund wird ein sehr einfaches Modell angenommen.
Zur Vebesserung der Ergebnisse sollte der Untergrund genauer untersucht werden und ein passenderes Modell gewählt werden.

Alle durchgeführten Methoden zur Bestimmung der Aktivierungsenergie $W$
liefern Werte im erwarteten Bereich von einem Elektronenvolt.
Die mit den verschiedenen Methoden berechneten Aktivierungsenergien weichen jedoch um mehrere Standardabweichungen voneinander ab.
Diese Abweichungen deuten auf systematische Fehler hin und müssen mit weiteren Messungen untersucht werden.

Die bestimmten Werte der Relaxationszeit $\tau_0$ sind schließlich nicht signifikant.
Die angegebenen Unsicherheiten von $\tau_0$ sind stark von der Aktivierungsenergie $W$ abhängig.
Bei einer Änderung von $W$ um einige \SI{100}{\milli\electronvolt} ändert
sich der Wert der Relaxationszeit bereits um etliche Größenordnungen, wie an den berechneten Relaxationszeiten zu sehen ist.
Änderungen der Aktivierungsenergie in diesem Bereich treten bereits auf, wenn der Wertebereich des Fits leicht verändert wird und einige Messpunkte mehr oder weniger betrachtet werden.
Die damit verbundenen Unsicherheiten machen eine Aussage über den wahren Wert von
$\tau_0$ unmöglich.

Das höhere Maximum in der Abbildung \ref{fig:plot2} im Bezug zum Maximum in der Abbildung \ref{fig:plot1}, lässt
sich dadurch erklären, dass für einen schnelleren Temperaturanstieg mehr Dipole gleichzeitig relaxieren.
Denn die Relaxationszeit ist exponentiell von der reziproken Temperatur abhängig.
Die Position des Maximums verschiebt sich außerdem für höhere Heizraten zu höheren Temperaturen, da der Temperaturanstieg
schneller abläuft, als dass die Dipole direkt folgen könnten und sich wieder statistisch verteilen.
